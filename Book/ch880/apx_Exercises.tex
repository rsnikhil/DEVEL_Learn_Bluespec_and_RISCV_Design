% -*- mode: fundamental -*-

% ****************************************************************

\chapter{Exercises}

\markboth{Ch \arabic{chapter}: Exercises (DRAFT)}{\copyrightnotice}

\setcounter{page}{1}
\renewcommand{\thepage}{\Alph{chapter}-\arabic{page}}

\label{apx_exercises}

% ----------------

NOTE: \fbox{
\begin{minipage}{5in}

\emph{This appendix may change are we improve or add new exercises.  Please
ensure you have the latest version!}

\end{minipage}}

\vspace{5ex}
% ----------------

% \footnotesize

% ****************************************************************

\hdivider

\section*{Chapter 3: {\BSV}: Overview of the BSV program structure}

% ================================================================

\subsection*{Ex-03-A-Hello-World}
\label{Ex-03-A-Hello-World}

(Refer Section~\ref{Sec_Hello_World}.)

Please see directory: \hm {\tt Exercises/Ex-03-A-Hello-World/} \\
and its README.

% ================================================================

\subsection*{Ex-03-B-Top-and-DUT}
\label{Ex-03-B-Top-and-DUT}

(Refer Section~\ref{Sec_Top_and_DUT}.)

Please see directory: \hm {\tt Exercises/Ex-03-B-Top-and-DUT/} \\
and its README.

% ================================================================

\subsection*{Ex-03-C-Module-and-Interface}
\label{Ex-03-C-Module-and-Interface}

(Refer Section~\ref{Sec_Module_and_Interface}.)

Please see directory: \hm {\tt Exercises/Ex-03-C-Module-and-Interface/} \\
and its README.

% ****************************************************************

\hdivider

\section*{Chapter 4: {\BSV}: Combinational circuits for the RISC-V Stage functions}

% ================================================================

\subsection*{Ex-04-A-Bit-Vectors: Bit-vectors}
\label{Ex-04-A-Bit-Vectors}

(Refer Sections~\ref{Sec_BSV_Bit_Vectors}-\ref{Sec_Dont_Care_Values}.)

Please see directory: \hm {\tt Exercises/Ex-04-A-Bit-Vectors/} \\
and its README.

% ================================================================

\subsection*{Ex-04-B-Bit-Vectors-Slicing: Slicing Bit-vectors}
\label{Ex-04-B-Bit-Vectors-Slicing}

(Refer Section~\ref{Bit_Vector_Ops}.)

Please see directory: \hm {\tt Exercises/Ex-04-B-Bit-Vectors-Slicing/} \\
and its README.

% ================================================================

\subsection*{Ex-04-C-Bit-Vectors-Operations: Operations on Bit-vectors}
\label{Ex-04-C-Bit-Vectors-Operations}

(Refer Section~\ref{BSV_functions}.)

Please see directory: \hm {\tt Exercises/Ex-04-C-Bit-Vectors-Operations/} \\
and its README.

% ================================================================

\subsection*{Ex-04-D-Decode-Imm: Functions to decode immediates}
\label{Ex-04-D-Decode-Imm}

(Refer Section~\ref{BSV_functions}.)

Please see directory: \hm {\tt Exercises/Ex-04-D-Decode-Imm/} \\
and its README.

% ================================================================

\subsection*{Ex-04-E-is-legal-XXX: Functions to check legality of instructions}
\label{Ex-04-E-is-legal-XXX}

(Refer Section~\ref{BSV_functions_2}.)

Please see directory: \hm {\tt Exercises/Ex-04-E-is-legal-XXX/} \\
and its README.

% ================================================================

\subsection*{Ex-04-F-FSM-Testbench: FSMs for testbenches}
\label{Ex-04-F-FSM-Testbench}

(Refer Section~\ref{BSV_small_testbench}.)

Please see directory: \hm {\tt Exercises/Ex-04-F-FSM-Testbench/}
\\ and its README.

% ================================================================

\subsection*{Ex-04-G-Enums-Muxes: Enums and Muxes}
\label{Ex-04-G-Enums-Muxes}

(Refer Sections~\ref{BSV_enum_types}-\ref{BSV_case_expressions}.)

Please see directory: \hm {\tt Exercises/Ex-04-G-Enums-Muxes/} \\
and its README.

% ****************************************************************

\hdivider

\section*{Chapter 5: Structs, tuples, memory requests and responses}

% ================================================================

\subsection*{Ex-05-A-Structs: structs}
\label{Ex-05-A-Structs}

Please see the directory: \hm {\tt Exercises/Ex-05-A-Structs/} \\
In this exercise we are going to use some actual Drum/Fife code;
specifically, these files:

\begin{tabbing}
\hmmmm \= \verb|Code/src_Common/Instr_Bits.bsv| \\
       \> \verb|Code/src_Common/Mem_Req_Rsp.bsv|
\end{tabbing}

\begin{itemize}

\item[(1)] In \verb|Makefile|, study and understand the new additions
    for this exercise.

    Hint: think about the ``\verb|import|'' statements in
    \verb|src_BSV/Top.bsv|.

    In order to build, you will likely have to edit the definition of
    \verb|BOOK_REPO| in the \verb|Makefile| to point at the correct
    directory in your environment.

    Study and understand the code in \verb|Top.bsv|.

    Compile and run the code using Verilator simulation or {\BLUESIM}
    simulation.  Observe and understand the behavior.

\item[(2)] Comment-out the \verb|import Instr_Bits| and \verb|import Mem_Req_Rsp|
    statements, one at a time, and try to recompile.

    Compilation will fail: in each case, observe and understand the error message.

\item[(3)] Comment-out the \verb|data| field definition in the struct definition.

    Recompile, and observe and understand the compiler message.

    Recompile and run.  What is the difference between the value
    printed for the \verb|data| field and what was printed earlier?

\item[(4)] Make two copies of the \verb|$display()| statement. In the
    two new copies replace:
    \begin{tabbing}
    \hm and \hm \= \kill
                \> {\tt fshow (mem\_req)} \\
    \hm with    \> {\tt mem\_req} \\
    \hm and     \> {\tt fshow\_Mem\_Req (mem\_req)} \\
    \hm respectively
    \end{tabbing}

    Note, \verb|fshow_Mem_Req()| is defined in the file
    {\tt Code/src\_Common/Mem\_Req\_Rsp.bsv}.

    Recompile and run. Observe and understand the differences in the
    outputs from the original and from the two new statements.

\item[(5)] In file \verb|Mem_Req_Rsp.bsv|, locate the function
    \verb|fshow_Mem_Req_Type()|.  Understand the code (see
    Section~\ref{Sec_Fmt_formatted_values} which explains the
    \verb|Fmt| type and the \verb|$format()| function).

    After the existing \verb|$display()| statement, add a line like this:

    \hmm \verb|$display ("mem_req.req_type = ",| \\
    \hmm \verb|          fshow_Mem_Req_Type (mem_req.req_type));|

    Recompile and run. Observe and understand the behavior.

    Compare the value printed for the \verb|req_type| field in the two
    \verb|$display()| statements.

\item[(6)] After the \verb|$display()| statement, add a line that
    updates the \verb|mem_req_type| field of the struct by assigning
    \verb|funct5_STORE| to the field (replacing its current value of
    \verb|funct5_LOAD)|. Then, replicate the \verb|$display()|
    statement to print the new value of the struct.

    Recompile and run (you should see output of original and new
    \verb|$display|).  Observe and understand the behavior.

\end{itemize}

% ================================================================

\subsection*{Ex-05-B-Mem-Req-Rsp: Memory Requests and Responses}
\label{Ex-05-B-Mem-Req-Rsp}

Please see the directory: \hm {\tt Exercises/Ex-05-B-Mem-Req-Rsp/} \\
In these exercises we will write a testbench that sends read requests
to a memory model and examines the memory responses.

We are going to use some additional actual Drum/Fife code;
specifically, these files:

\begin{tabbing}
\hmmmm \= \verb|Code/src_Common/Utils.bsv| \\
       \> \hmmmm \= (for \verb|Initial_Params| struct definition) \\
       \> \verb|Code/src_Top/Mems_Devices.bsv| \\
       \>        \> (memory/devices model top-level) \\
       \> \verb|Code/src_Top/C_Mems_Devices.c| \\
       \>        \> (memory model implementation in C) \\
       \> \verb|Code/src_Top/UART_model.c| \\
       \>        \> (UART model implementation in C) \\
       \> \verb|Code/vendor/bluespec_BSV_Additional_Libs/Semi_FIFOF.bsv| \\
       \>        \> (for connections to memory model)
\end{tabbing}

\begin{itemize}

\item[(1)] In \verb|Makefile|, study and understand the new additions
    for this exercise.

    Hint 1: think about the \verb|import| statements in \verb|src_BSV/Top.bsv|.

    Hint 2: In file \verb|Code/src_Top/Mems_Devices.bsv| there are
        some \verb|import "BDPI"| statements that refer to C functions
        that are defined in the C files mentioned above.

    In order to build, you will likely have to edit the definition of
    \verb|BOOK_REPO| in the \verb|Makefile| to point at the correct
    directory in your environment.

    Study and understand the code in \verb|Top.bsv|.

    When the program is run, the statement: \\
    \hmm \verb|mems_devices.init (init_params);| \\
    will initialize the memory model and load it with data from
    \verb|test.memhex32|.

    Study and understand the contents of file \verb|test.memhex32|.
    It contains 16 instructions from the \verb|Hello World!| program,
    starting at address 0x\_8000\_0000.

    Study the file: \\
    \hmm \verb|Tools/Hello_World_Example_Code/hello.RV32.bare.objdump| \\
    and locate these 16 instructions. There you can see the assembly
    language for these instructions.

    Compile and run the code using Verilator simulation or {\BLUESIM}
    simulation.  Observe and understand the behavior.

\item[(2)] There are two \verb|Top.bsv| files visible to the {\bsc} compiler:
    \begin{tightlist}
    \item In this exercise: \verb|./src_BSV/Top.bsv)|
    \item In Drum/Fife:     \verb|Code/src_Top/Top.bsv|
    \end{tightlist}
    When we compile, why does {\bsc} pick the former and not the latter?

\item[(3)] In \verb|./src_BSV/Top.bsv|, in module \verb|mkTop|, just
    before \verb|mkAutoFSM|, define a function to encapsulate and
    generalize the \verb|a_req| action:

    {\footnotesize
    \begin{Verbatim}[frame=single, numbers=left]
      function Action a_req (Mem_Req_Type t,
                             Mem_Req_Size s,
                             Bit #(64)    a,    // addr
                             Bit #(64)    d,    // data
                             Bit #(64)    i);   // inum
        ... body is a_req action with above arguments instead of constants ...
      endfunction
    \end{Verbatim}
    }

    Similarly, define a function to encapsulate and generalize the
    \verb|a_rsp| action:

    {\footnotesize
    \begin{Verbatim}[frame=single, numbers=left]
      function Action a_rsp ();
        ... body is a_rsp action ...
      endfunction
    \end{Verbatim}
    }

    Replace the original \verb|a_req| and \verb|a_rsp| actions with
    calls to these functions.  Note, you won't need
    '\verb|action-endaction|' brackets around these 1-line calls.

    Recompile and run and verify that the output is the same as before.

\item[(4)] Add an additional memory request and response for the next
    instruction ({\tt addr} = 0x\_8000\_0004, {\tt inum} = 2):

    {\footnotesize
    \begin{Verbatim}[frame=single, numbers=left]
        // Original addr
        a_req (..., 'h_8000_0000, ..., 1);
        a_rsp;    
        // Next instruction addr
        a_req (..., 'h_8000_0004, ..., 2);
        a_rsp;    
    \end{Verbatim}
    }

    Recompile and run and verify that the output is as expected
    (matches contents of \verb|test.memhex32|).

\item[(5)] Add more calls to \verb|a_req()|/\verb|a_rsp()| (or replace
    the earlier ones):

    \begin{itemize}

    \item Vary the \verb|Mem_Req_Size| argument to be, variously, 1, 2
        and 4 bytes.

    \item For the 2 and 4 bytes sizes, try both aligned and unaligned
        addresses.
    \end{itemize}

    In each case, recompile and run, and observe and understand the
    behavior.

\item[(6)] In each run so far, observe that the memory system
    announces the range of addresses that it implements for the memory
    model:

    {\footnotesize
    \begin{Verbatim}[frame=single, numbers=left]
    INFO: c_mems_devices_init
      Mem system model
       ADDR_BASE_MEM:  0x80000000 SIZEB_MEM:  0x10000000 (268435456) bytes
    \end{Verbatim}
    }

    Note that the memory size is much larger than the contents of
    \verb|test.memhex32|, which means that most of the locations
    remain uninitialized.

    Add more calls to \verb|a_req()|/\verb|a_rsp()| (or replace the
    earlier ones):

    \begin{itemize}
    \item Try a LOAD from address 0x\_8000\_0040

        (This address is implemented by the memory model, but is not
         defined in \verb|test.memhex32|).

    \item Try a LOAD from address 0x\_9000\_0000

        (This address is not implemented by the memory model).
    \end{itemize}

    In each case, recompile and run, and observe and understand the output.

\item[(7)] Add more calls to \verb|a_req()|/\verb|a_rsp()| (or replace
    the earlier ones):

    \begin{itemize}
    \item Do a STORE to some memory location.

    \item Then do a LOAD to read it back, and verify that the loaded
          value is indeed what we stored.

    \item Do LOADs with addresses on either side of the STORE address/size and
          verify that the STORE did not disturb those locations.
    \end{itemize}

    In each case, recompile and run, and observe and understand the output.

\item[(8)] In each run so far, observe that the memory system
    announces the range of MMIO addresses that it implements for the
    UART model:

    {\footnotesize
    \begin{Verbatim}[frame=single, numbers=left]
    INFO: c_mems_devices_init
       ...
       ADDR_BASE_UART: 0x60100000 SIZEB_UART: 0x00001000 (4096) bytes
    \end{Verbatim}
    }


    The UART models a standard 16550 UART, which has several
    memory-mapped registers, but for this test we will only use
    0x\_6010\_0000 which is the ``transmit buffer''.  Every time we
    write a byte to the transmit buffer, it gets ``transmitted'' and
    displayed on the screen.

    Add more calls to \verb|a_req()|/\verb|a_rsp()| (or replace the
    earlier ones): do a series of \verb|a_req()| calls, one for each
    character in ``\verb|Hello World!\r\n|''.  Each call should be for
    a 1-byte STORE, to address 0x\_6010\_0000, with data being the
    ASCII hex code of that character.

    Recompile and run, and observe and understand the behavior.

    [Not so easy!] Why is ``\verb|Hello World!\r\n|'' all printed
    together, instead of each character being printed as soon as we do
    the corresponding STORE?  \emph{Hint:} See {\tt "// Write the
    char to the output line buffer"} and lines 526-535 that follow, in
    \verb|UART_Model.c|.

\item[(9)] When printing ``\verb|Hello World!\r\n|'' ({\ie} 14
    characters), we alternated 14 \verb|a_req()|'s with 14
    \verb|a_rsp()|'s, like this:

    \begin{tabbing}\footnotesize\tt
    a\_req(); a\_rsp(); ...; a\_req(); a\_rsp();    // 14 pairs
    \end{tabbing}

    Reorganize the code so that we first have all 14 \verb|a_req()|'s
    followed by all 14 \verb|a_rsp()|'s:

    \begin{tabbing}\footnotesize\tt
    a\_req(); ...; a\_req();    // 14 requests
    a\_rsp; ...; a\_rsp();      // 14 responses
    \end{tabbing}

    Recompile and run, and observe and understand the behavior.

    Hint: the path in and out of memory is pipelined (so we can send
    in multiple requests before retrieving a response).  But it has a
    finite depth, so when it ``fills up'', the pipeline gets stuck; it
    cannot accept any more requests until we drain some responses.

\item[(10)] Reorganize the code from this:

    \begin{tabbing}\footnotesize\tt
    a\_req(); ...; a\_req();    // 14 requests
    a\_rsp; ...; a\_rsp();      // 14 responses
    \end{tabbing}

    to this:

    \begin{tabbing}\footnotesize\tt
    par
       seq a\_req(); ...; a\_req(); endseq
       seq a\_rsp; ...; a\_rsp();   endseq
    endpar
    \end{tabbing}

    The '\verb|par|-\verb|endpar|' block allows its two components,
    each of which is a sequence of 14 actions, to run \emph{concurrently}.

    Recompile and run, and observe and understand the behavior.

\item[(11)] Study the C files to see how the memory model and UART
    model are implemented.

\end{itemize}

% ****************************************************************

\hdivider

\section*{Chapter 6: Core functions}

% ================================================================

\subsection*{Ex-06-A-Fetch}
\label{Ex-06-A-Fetch}

Please see the directory: \hm {\tt Exercises/Ex-06-A-Fetch/} \\
This is a minimal ``unit testbench'' for \verb|fn_Fetch()|.  It
instantiates the memory model used by Drum and Fife.  The testbench
uses \verb|fn_Fetch()| to generate memory requests for simulating
instruction-fetch.

As in Ex-05-B-Mem-Req-Rsp, the file 'test.memhex32' contains 16
instructions from the \verb|Hello World!| program starting at address
0x\_8000\_0000.  When the program is run, the statement:

{\footnotesize
\begin{Verbatim}[frame=single]
    mems_devices.init (init_params);
\end{Verbatim}
}

will initialize the memory model and load it with data from
\verb|test.memhex32|.

\begin{itemize}

\item[(1)] Exercise baseline

    Study and understand the code in \verb|Top.bsv|.

    Compile and run the code using Verilator simulation or {\BLUESIM}
    simulation.  Observe and understand the behavior.

\item[(2)] Add a few more calls to \verb|a_req()| and \verb|a_rsp()|
    to fetch a few more instructions from successive addresses
    0x\_8000\_0004, ...0008, ... {\etc}.

    Recompile and run.  Observe and understand the behavior.

\item[(3)] Memory access errors

    Add a few more calls to \verb|a_req()| and \verb|a_rsp()| to fetch
    a few more instructions from:

    \begin{tightlist}
    \item An unimplemented address ({\eg} 0x\_9\_0000\_0000)
    \item A misaligned address ({\eg} 0x\_8000\_0002)
    \end{tightlist}

    Recompile and run. Observe and understand the behavior.

\item[(4)] Scope of identifiers

    In \verb|Top.bsv|, the functions \verb|a_req()| and \verb|a_rsp()|
    are defined inside module \verb|mkTop| after the instantiations of
    FIFOs \verb|f_req| and \verb|f_rsp|.

    Move the instantiations of FIFOs \verb|f_req| and \verb|f_rsp| so
    that they come \emph{after} the two functions.

    Recompile and run.  Observe and understand the behavior.

\item[(5)] Interfaces can be passed as function arguments

    Move the two function definitions \verb|a_req()| and
    \verb|a_rsp()| to be \emph{outside} and \emph{before} module
    \verb|mkTop|'s definition, as follows:

    \begin{itemize}
    \item Add an argument to \verb|a_req()|:

          {\footnotesize
\begin{Verbatim}[frame=single]
    function Action a_req (... inum, FIFOF #(Mem_Reqs) f_reqs);
\end{Verbatim}
          }
          and move this function out of the module, before the module,
          at the top-level of the package.

    \item Add an argument to \verb|a_rsp()|:
          {\footnotesize
\begin{Verbatim}[frame=single]
    function Action a_rsp (FIFOF #(Mem_Rsps) f_rsps);
\end{Verbatim}
          }

          and move this function out of the module, before the module,
          at the top-level of the package.
    \end{itemize}

    In \verb|mkAutoFSM|,
    \begin{tightlist}
    \item in each invocation of \verb|a_req()|, add \verb|f_reqs| as an argument.
    \item in each invocation of \verb|a_rsp()|, add \verb|f_rsps| as an argument.
    \end{tightlist}

    Recompile and run.  Verify that the behavior is the same as before.

\end{itemize}

This testbench performs a few so-called ``directed tests'' for
\verb|fn_Fetch()|.  By generalizing the FSM to be a loop and to
generate random addresses, the testbench can be made into a stronger
``unit testbench'' that tests many more possibilities.  See
Section~``3.8.1~LFSR'' in the {\bsc} Libraries Reference Guide for
facilities to generate pseudo-random numbers.

% ================================================================

\subsection*{Ex-06-B-Decode}
\label{Ex-06-B-Decode}

Please see the directory: \hm {\tt Exercises/Ex-06-B-Decode/} \\
This is a minimal unit testbench for \verb|fn_Decode()|.  It builds on
the previous exercise, using an output of \verb|fn_Fetch()| and the
memory response as its inputs.

\begin{itemize}

\item[(1)] Exercise baseline.

    Study and understand the code in \verb|Top.bsv|.  It will be
    useful to ``diff'':
    \begin{tightlist}
    \item \verb|Ex-06-A-Fetch/src_BSV/Top.bsv|
    \item \verb|Ex-06-B-Decode/src_BSV/Top.bsv|
    \end{tightlist}
    to see what has changed.

    Compile and run the code using Verilator simulation or {\BLUESIM}
    simulation.  Observe and understand the behavior.

    Verify that the output of \verb|fn_Decode()| is correct for the
    fetched instruction.  (Note, the program fetches a 32-bit
    instruction from address 0x\_8000\_0000 (the particular
    instruction can be seen in file \verb|test.memhex32|).  Locate
    this instruction in the ``objdump'' file:

    \hmm {\tt Code/Tools/Hello\_World\_Example\_Code/hello.RV32.bare.objdump }

    which shows the assembly language code for the instruction.)

\item[(2)] Repeat, for a few more instructions at other addresses.

\item[(3)] Repeat, for a few more instructions at:
    \begin{tightlist}
    \item An unimplemented address ({\eg} 0x\_9\_0000\_0000)
    \item A misaligned address ({\eg} 0x\_8000\_0002)
    \end{tightlist}

\item[(4)] Create a new \verb|test.memhex32| file with just 4 bytes of
    data at 0x\_8000\_0000.  Edit the testbench to fetch just this one
    instruction.

    Repeat:
    \begin{itemize}

    \item Edit the 4 bytes to represent a different instruction that
      will take the test program into a different branch of the
      if-then-else's in \verb|fn_Decode()|.

    \item Recompile and run. Verify that the output of
      \verb|fn_Decode| is as expected.

    \end{itemize}

\end{itemize}

% ================================================================

\subsection*{Ex-06-C-Dispatch}
\label{Ex-06-C-Dispatch}

Please see directory: \hm {\tt Exercises/Ex-06-C-Dispatch/} \\
This is a minimal unit testbench for \verb|fn_Dispatch()|.  It builds
on the previous exercise, using the output of \verb|fn_Decode()| as
its input.

\begin{itemize}

\item[(1)] Exercise baseline

    Study and understand the code in \verb|Top.bsv|. It will be useful
    to ``diff'':
    \begin{tightlist}
    \item \verb|Ex-06-B-Decode/src_BSV/Top.bsv|
    \item \verb|Ex-06-C-Dispatch/src_BSV/Top.bsv|
    \end{tightlist}
    to see what has changed.

    Compile and run the code using Verilator simulation or {\BLUESIM}
    simulation.  Observe and understand the behavior.

    Verify that the output of \verb|fn_Dispatch()| is correct for the
    fetched instruction.  (Note, the program fetches a 32-bit
    instruction from address 0x\_8000\_0000 (the particular
    instruction can be seen in file \verb|test.memhex32|).  Locate
    this instruction in the ``objdump'' file:

    \hmm {\tt Code/Tools/Hello\_World\_Example\_Code/hello.RV32.bare.objdump }

    which shows the assembly language code for the instruction.)

\item[(2)] The output of \verb|fn_Dispatch()| (type
    \verb|Result_Dispatch|) has four sub-structs:
    {\footnotesize\tt
    \begin{tabbing}
    \hm \= RR\_to\_EX\_Control \hm \= to\_EX\_Control; \kill
        \> RR\_to\_Retire          \> to\_Retire; \\
        \> RR\_to\_EX\_Control     \> to\_EX\_Control; \\
        \> RR\_to\_EX              \> to\_EX; \\
        \> Mem\_Req                \> to\_EX\_DMem;
    \end{tabbing}}

    The \verb|to_Retire| sub-struct has a field \verb|exec_tag| of type
    \verb|Exec_Tag|.
    \begin{tabbing}
    The {\tt to\_EX\_Control} \= sub-struct is only valid when {\tt exec\_tag == EXEC\_TAG\_CONTROL}. \\
    The {\tt to\_EX}          \> sub-struct is only valid when {\tt exec\_tag == EXEC\_TAG\_INT}. \\
    The {\tt to\_EX\_DMem}    \> sub-struct is only valid when {\tt exec\_tag == EXEC\_TAG\_DMEM}.
    \end{tabbing}

    The code currently \verb|$display|'s all the sub-structs, whether
    valid or not.  Replace this code with if-then-else's so that
    sub-structs are only \verb|$display|'ed when they are valid.

    Recompile and run to verify the code is running correctly.

\item[(3)] Repeat, for a few more instructions at other addresses.

\item[(4)] Repeat, for a few more instructions at:
    \begin{tightlist}
    \item An unimplemented address ({\eg} 0x\_9\_0000\_0000)
    \item A misaligned address ({\eg} 0x\_8000\_0002)
    \end{tightlist}

\item[(5)] Create a new \verb|test.memhex32| file with just 4 bytes of
    data at 0x\_8000\_0000.  Edit the testbench to fetch just this one
    instruction.

    Repeat:
    \begin{itemize}

    \item Edit the 4 bytes to represent a different instruction that
      will take the test program into a different branch of the
      if-then-else's in \verb|fn_Dispatch()|.

    \item Recompile and run. Verify that the output of
          \verb|fn_Dispatch()| is as expected.

    \end{itemize}

\end{itemize}

% ================================================================

\subsection*{Ex-06-D-Ex-Control}
\label{Ex-06-D-Ex-Control}

Consider {\tt fn\_EX\_Control} in Section~\ref{Sec_fn_EX_Control}.

\begin{itemize}

\item[(1)] The BRANCH, JAL and JALR clauses set the {\tt exception}
    field to true if the next PC is not aligned.

    What would happen if we did not set the exception field here?

    (See ``Section 2.5 Control Transfer Instructions'' in the
     Unprivileged ISA specification document
     \cite{RISCV_Unpriv_2019_12_13} for a discussion of why we set it
     here.)

\item[(2)] Prove (informally) that the three-way if-then-else in
    \verb|Fn_EX_Control| will catch all cases, {\ie} that we never
    need a final ``\verb|else|'' clause.  This requires reviewing
    \verb|fn_Decode| and tracking the flow of information through the
    Register-Read-and-Dispatch stage (including \verb|fn_Dispatch)|,
    into \verb|fn_Control|.

\item[(3)] In \verb|Fn_EX_Control|, can we change the final ``\verb|else if|'':

    {\footnotesize
    \begin{Verbatim}[frame=single]
   else if (is_JALR (instr)) begin
    \end{Verbatim}
    }

    into a simple ``\verb|else|'', {\ie} omit the the \verb|is_JALR|
    check?

    {\footnotesize
    \begin{Verbatim}[frame=single]
   else begin
    \end{Verbatim}
    }

    What might be the hardware implication of such a change?

\item[(4)] In the final section of {\tt fn\_EX\_Control}, why do we
    have: \verb|data: x.fallthrough_pc|?

    \emph{Hint}: review the semantics of JAL and JALR instructions.

    What if this is a BRANCH instruction?

    What if this is a JAL or JALR instruction with \verb|rd| = 0?
\end{itemize}

Please see the directory: \hm {\tt Exercises/Ex-06-D-Ex-Control/} \\
This is a minimal ``unit testbench'' for \verb|fn_Ex_Control()|.  It
instantiates the memory model used by Drum and Fife.  It prepares
input data for \verb|fn_Ex_Control()| using \verb|fn_Fetch()|,
\verb|fn_Decode()|, and \verb|fn_Dispatch()|.

\begin{itemize}

\item[(5)] Study and understand the code in \verb|Top.bsv|. It will be
    useful to ``diff'':

    \begin{tightlist}
    \item \verb|Ex-06-C-Dispatch/src_BSV/Top.bsv|
    \item \verb|Ex-06-D-Ex-Control/src_BSV/Top.bsv|
    \end{tightlist}
    to see what has changed.

    Compile and run the code using Verilator simulation or {\BLUESIM}
    simulation.  Observe and understand the behavior.

    Verify that the output of \verb|fn_EX_Control()| is correct for
    the fetched instructions. (Note, the program fetches two 32-bit
    instructions from addresses 0x\_8000\_0000 and 0x\_8000\_003c (the
    particular instruction can be seen in file \verb|test.memhex32|).
    Locate this instruction in the ``objdump'' file:

    \hmm {\tt Code/Tools/Hello\_World\_Example\_Code/hello.RV32.bare.objdump }

    which shows the assembly language code for the instructions.)
    The first of these two instructions is not a control
    instruction, and the second is a control instruction (BNE).

    For the two BNE instructions, we invoke \verb|a_rsp (0,0)| and
    \verb|a_rsp (0,1)|.  The two arguments represent the rs1 and rs2
    values.  BNE should be False for the first one, True for the
    second (so the next PC will be different for the two cases).

\item[(6)] Create a new \verb|test.memhex32| file with just 4 bytes of
    data at 0x\_8000\_0000.  Edit the testbench to fetch just this one
    instruction.

    Repeat:
    \begin{itemize}

    \item Edit the 4 bytes to represent a different control instruction
          (try BEQ, BNE, BLT, BGE, BLTU, BGEU, JAL, and JALR).

    \item Recompile and run. Verify that the output of
          \verb|fn_EX_Control()| is as expected.

    \end{itemize}

\end{itemize}

% ================================================================

\subsection*{Ex-06-E-Ex-Int}
\label{Ex-06-E-Ex-Int}

Consider {\tt fn\_IALU} in Section~\ref{Sec_fn_EX_Int}.

\begin{itemize}

\item[(1)] Lines 16-53 contain two if-statements without ``else''
    clauses.  We could instead have written it this way:

    {\footnotesize
    \begin{Verbatim}[frame=single, numbers=left]
   Bit #(XLEN) y = 0;
   if (opcode == opcode_OP) begin
      ...
      ... y = ...
   end
   else if (opcode == opcode_OP_IMM) begin
      ...
      ... y = ...
   end
   return y;
    \end{Verbatim}
    }

    Discuss the hardware tradeoffs between writing it in these two
    ways.  {\emph Hints:} Consider:

    \begin{tightlist}

      \item Sequentiality of if-then-else.

      \item Ability (or not) to prove exhaustiveness of conditions in
          nested if-then-else.

      \item Ability (or not) to prove mutual-exclusivity of conditions
          in nested if-then-else.

      \item Discussion in Section~\ref{Sec_MUXes} on parallel and
        sequential multiplexers (mux).  Note: in our code, we have
        explicitly coded a parallel mux.

    \end{tightlist}

\item[(2)] Justify the presence or absence of the ``{\tt pack}''
    operator in each ``case'' arm in {\tt fn\_IALU}.

\item[(3)] The RISC-V ISA has ADD, SUB and ADDI instructions, but no
    SUBI instruction.  Why not?

\item[(4)] Suppose we want to extend {\tt fn\_IALU} so it also works
    when XLEN=64 ({\ie} for RV64I).  What needs to change to
    accommodate this?

    \emph{Hint:} it only matters in the shift-amount of the shift
    instructions, where the shift-amount can be 6-bits wide instead of
    5-bits (allowing a maximum of 63-bit shifts instead of 31 bits).

\end{itemize}

Please see the directory: \hm {\tt Exercises/Ex-06-E-Ex-Int/} \\
This is a minimal ``unit testbench'' for \verb|fn_Ex_Int()|.  It
instantiates the memory model used by Drum and Fife.  It prepares
input data for \verb|fn_Ex_Int()| using \verb|fn_Fetch()|,
\verb|fn_Decode()|, and \verb|fn_Dispatch()|.

\begin{itemize}

\item[(5)] Study and understand the code in \verb|Top.bsv|. It will be
    useful to 'diff':
    \begin{tightlist}
    \item \verb|Ex-06-C-Dispatch/src_BSV/Top.bsv|
    \item \verb|Ex-06-D-Ex-Int/src_BSV/Top.bsv|
    \end{tightlist}
    to see what has changed.

    Compile and run the code using Verilator simulation or {\BLUESIM}
    simulation.  Observe and understand the behavior.

    Verify that the output of \verb|fn_EX_Int()| is correct for the
    fetched instruction. (Note, the program fetches a 32-bit
    instruction from address 0x\_8000\_0000 (the particular
    instruction can be seen in file \verb|test.memhex32|).  Locate
    this instruction in the ``objdump'' file:

    \hmm {\tt Code/Tools/Hello\_World\_Example\_Code/hello.RV32.bare.objdump }

    which shows the assembly language code for the instruction.)

    For the instruction, we invoke \verb|a_rsp (0,0)|.  The two
    arguments represent the rs1 and rs2 values (which may or may not
    be relevant for different int opcodes).

\item[(6)] Create a \verb|test.memhex32| file with just 4 bytes of
    data at 0x\_8000\_0000.  Edit the testbench to fetch just this one
    instruction.

    Repeat:
    \begin{itemize}

    \item Edit the 4 bytes to represent a different integer instruction
        (try various Int opcodes, with different rs1 and rs2 values).

    \item Recompile and run. Verify that the output of
          \verb|fn_EX_Int()| is as expected.

    \end{itemize}

\end{itemize}

% ****************************************************************

\hdivider

\section*{Chapter 7: BSV: Modules and Interfaces:
Registers, Register Files and FIFOs}

% ================================================================

\subsection*{Ex-07-A-Interface-Transformers}
\label{Ex-07-A-Interface-Transformers}

(Please refer to Section~\ref{Sec_interface_transfomers}.)

Write a similar function to view the input side of a \verb|FIFOF|
interface as a {\tt FIFOF\_I} interface.

% ================================================================

\subsection*{Ex-07-B-Polymorphic-Types}
\label{Ex-07-B-Polymorphic-Types}

(Please refer to Section~\ref{Sec_Polymorphic_Types}.)

The types {\tt Mem\_Req} and {\tt Mem\_Rsp}
(Sections~\ref{Sec_Mem_Req} and \ref{Sec_Mem_Rsp}) are monomorphic.
Write polymorphic versions of these types that are parameterized on
{\tt xlen} for the {\tt addr} field.

% ================================================================

\subsection*{Ex-07-C-Synthesizable-Modules}
\label{Ex-07-C-Synthesizable-Modules}

(Please refer Section~\ref{Sec_Polymorphic_Types_and_Synthesizability}.)

\begin{itemize}

\item[(1)] In {\tt mkRISCV\_GPRs\_V} replace the explicit type declaration
      of {\tt ifc} with the {\tt let} keyword.

\item[(2)] Write a monomorphic wrapper for the BSV library {\tt mkFIFOF}
      module that specializes it into a FIFO that only carries {\tt
      Bool} values.  Add an annotation so the wrapper becomes a
      separate Verilog module.  Compile it and study the generated
      Verilog.

\end{itemize}

% ****************************************************************

\hdivider

\section*{Chapter 8: RISC-V: Modules for GPRs and CSRs}

% ================================================================

\subsection*{Ex-08-A: GPR {\tt x0} is constant zero}
\label{Ex-08-A-GPR-x0}

(Please refer Section~\ref{Sec_RISCV_regfile}.)

Why did the RISC-V ISA designers specify that GPR {\tt x0} is constant zero?

\emph{Hint:} Many computations require a constant zero value.  For
example, in C, a {\tt for}-loop iterating over an array requires an
intitial index of zero.  Think about how you would produce a value
zero in a register without using {\tt x0}.

Why not reserve more GPRs for other common constant values (such as ``1'')?

% ================================================================

\subsection*{Ex-08-B: GPR {\tt x0} as a destination register}
\label{Ex-08-B-GPR-x0-as-Rd}

(Please refer Section~\ref{Sec_RISCV_regfile}.)

Why would one use {\tt x0} as a destination register ({\tt Rd} field of an instruction)?

\emph{Hint:}
\begin{itemize}

 \item Think about JUMP instructions where we are not interested in
       saving a "return address" (unconditional jumps, returns from
       subroutines, tail-recursion, continuation-passing-style, ...

 \item Think about CSR and AMO instructions where we are not
       interested in the value returned to a register.

\end{itemize}

% ================================================================

\subsection*{Ex-08-C: GPR Register Files}
\label{Ex-08-C-GPR-Register-Files}

(Please refer Section~\ref{Sec_RISCV_regfile}.)

\begin{itemize}

\item[(1)] In {\tt mkGPRs} we write the value when \emph{j} is zero, but we
      ignore it on reads.  Write a variant where, in {\tt write\_rd},
      we alway write 0 when the index {\tt rd} is zero, and the read
      methods no longer check if {\tt rs1} or {\tt rs2} are 0.

      In this variant, what happens if we try to read {\tt x0} before
      it is written for the first time?

      Compare the circuitry generated in the original and in the
      variant.  Why might we choose one over the other?

\item[(2)] In {\tt mkGPRs} suppose we use {\tt mkRegFile(1,31)} instead of
      {\tt mkRegFileFull}.  What needs to change to accommodate this?

\end{itemize}

% ================================================================

\subsection*{Ex-08-D: CSRs}
\label{Ex-08-D-CSRs}

(Please refer Section~\ref{Sec_RISCV_CSRs}.)

Write a testbench module that instantiates {\tt mkCSRs} and tests its
interface methods.

\begin{itemize}

 \item Invoke the {\tt mav\_csrrxx()} method to write some CSRs and
       read them back and print the results.

   \begin{itemize}
     \item Start by testing {\tt mtvec}, {\tt mepc}, {\tt mcause}, and {\tt mtval}.

     \item Try reading and writing to a CSR address that is illegal (unimplemented CSR).

     \item Try reading and writing to a CSR that is read-only ({\eg} {\tt MVENDORID}).
   \end{itemize}

 \item Use {\tt mav\_csrrxx()} to write {\tt mtvec}; invoke {\tt
       mav\_exception} method and check that it returns what was
       written in {\tt mtvec}.  Use {\tt mav\_csrrxx()} to read {\tt
       mepc}, {\tt mcause} and {\tt mtval} and check that they are
       correct.

 \item Use {\tt mav\_csrrxx()} to write {\tt mepc}; invoke {\tt
       read\_epc} and check that we get the correct value.

 \item Invoke {\tt ma\_incr\_instret}, then use {\tt mav\_csrrxx()} to
       read CSRs MINSTRET and MINSTRETH and check that we get the
       correct value.

 \item Invoke {\tt set\_TIME}, then use {\tt mav\_csrrxx()} to
       read CSRs TIME and TIMEH and check that we get the
       correct value.

\end{itemize}

[Advanced; optional] Why do we have CSRs TIME and TIMEH, but not MTIME
and MTIMEH? \\
\emph{Hint:} Read Section 3.2.1 in the Privileged ISA
specification, as a starting point.

% ****************************************************************

\hdivider

\section*{Chapter 10: RISC-V: The Drum CPU}

% ================================================================

\subsection*{Ex-10-A: Drum initialization}
\label{Ex-10-A-Drum-initialization}

(Please refer Section~\ref{Sec_Drum_CPU_module_behavior}.)

What might happen if we omitted the ``{\tt await!(rg\_running)}''
statement in the Drum CPU? (Try it in simulation!)

\emph{Hint:} The FSM may start running before the PC has been initialized ...

% ****************************************************************

\hdivider

\section*{Chapter 11: {\BSV}: Verifying {\BSV} designs}

% ================================================================

\subsection*{Ex-11-A: {\tt FShow}}
\label{Ex-11-A-FShow}

(Please refer Section~\ref{sec_FShow}.)

Write a small test program that constructs a struct value (of type
{\tt Decode\_to\_RR}, say), and then prints it in two different ways:

{\footnotesize
\begin{Verbatim}[frame=single]
   Decode_to_RR y = ... construct a Decode_to_RR struct value ...

   $display ("y         is ", y);
   $display ("fshow (y) is ", fshow (y));
\end{Verbatim}
}

Observe and comment on the difference.

% ****************************************************************

\hdivider

\section*{Chapter 13: {\BSV}: Rules and their Semantics}

% ================================================================

\subsection*{Ex-13-A: All {\tt Action}s in a rule are simultaneous}
\label{Ex-13-A-Simultaneous-Actions}

(Please refer Section~\ref{Sec_Single_Rule_Semantics}.)

Consider the following two alternative ways of writing a rule (that
differ only in the order of the rule-body statements):

\begin{center}
\begin{minipage}{0.4\textwidth}
 {\footnotesize
 \begin{Verbatim}[frame=single, label=BSV]
   rule rl_r1 (... condition ...);
    x <= y + 1;
    y <= x + 2;
   endrule
 \end{Verbatim}
 }
\end{minipage}
\hmm
\begin{minipage}{0.4\textwidth}
 {\footnotesize
 \begin{Verbatim}[frame=single,label=BSV]
   rule rl_r2 (... condition ...);
    y <= x + 2;
    x <= y + 1;
   endrule
 \end{Verbatim}
 }
\end{minipage}
\end{center}

Sketch the semantic view (and possible hardware) for the two
alternatives.  Is there any semantic difference between the two rules?

If the values of registers \verb|x| and \verb|y| are 10 and 20
respectively, what are their values after the rule fires once?  What
would the answer(s) be with the following similar-looking C
statements?

\begin{center}
\begin{minipage}{0.4\textwidth}
 {\footnotesize
 \begin{Verbatim}[frame=single, label=C/C++]
{
    x = y + 1;
    y = x + 2;
}
 \end{Verbatim}
 }
\end{minipage}
\hmm
\begin{minipage}{0.4\textwidth}
 {\footnotesize
 \begin{Verbatim}[frame=single, label=C/C++]
{
    y = x + 2;
    x = y + 1;
}
 \end{Verbatim}
 }
\end{minipage}
\end{center}

% ****************************************************************

\hdivider

\section*{Chapter 14: RISC-V: the Drum unpipelined CPU (using Rules instead of {\tt StmtFSM})}

% ================================================================

\subsection*{Ex-14-A: Optimizing Trum Rules}
\label{Ex-14-A-Optimizing-Drum-Rules}

(Please refer Section~\ref{Sec_Optimiizing_Drum_Rules}.)

Implement some or all the ideas in
Section~\ref{Sec_Optimiizing_Drum_Rules} and measure the results
(circuit size, achievable clock speed, application program speed).

% ================================================================

% ****************************************************************

\hdivider

\section*{Chapter 15: RISC-V: the Fife Pipelined CPU: Principles}

% ================================================================

\subsection*{Ex-15-A: Bit-width of {\tt rg\_epoch}}
\label{Ex-15-A-Epoch-Bitwidth}

(Please refer Section~\ref{Sec_Epochs}.)

We have described \verb|rg_epoch| as a counter that is incremented on
each recognition of a misprediction.  If the register contents have
type \verb|Bit #(n)|, then this will wrap-around to 0 after $2^n$
increments.  Is this a problem?

If \verb|rg_epoch| contains a \verb|Bit #(n)| value, how small can $n$ be?

% ================================================================

\subsection*{Ex-15-B: Hazards}
\label{Ex-15-B-Hazards}

(Please refer Section~\ref{Sec_Scoreboards}.)

For two consecutive instructions $I_1$ and $I_2$,
  \begin{tightlist}
    \item $I_1$ may want to write register $x_7$ and $I_2$ may want to read $x_7$,
    \item $I_1$ may want to write register $x_7$ and $I_2$ may write to write $x_7$,
    \item $I_1$ may want to read register $x_7$ and $I_2$ may want to read $x_7$,
    \item $I_1$ may want to read register $x_7$ and $I_2$ may want to write $x_7$.
  \end{tightlist}
Above, we motivated scoreboards with the first scenario.  What about
the other three scenarios?

% ================================================================

\subsection*{Ex-15-C: Write-Write Hazards}
\label{Ex-15-C-Write-Write-Hazards}

(Please refer Section~\ref{Sec_Scoreboards}.)

Write-write hazards can be treated just like read-after-write hazards.
Alternatively the 1-bit in the scoreboard for a register (say $x_7$)
can be generalized into an $n$ bit up/downcounter, indicating the
number of instructions that have been allowed into Execute pipelines
that intend to write $x_7$.  The Retire stage decrements this counter;
the Register-Read stage stalls an instruction if these counters (for
its input registers) are non-zero; and the Register-Read stage
increments this counter for an instruction's destination register.

Implement a scoreboard module with this scheme.  What should happen if
a counter reaches its maximum value?  How many bits should each
counter have?

% ================================================================

\subsection*{Ex-15-D: Write-Write ordering}
\label{Ex-15-D-Write-Write-ordering}

(Please refer Section~\ref{Sec_Scoreboards}.)

In the counter-based scoreboard of the previous exercise, if there are
multiple instructions in the Execute stages that intend to write
$x_7$, in what order can those writes occur?  What would be the
consequences of a wrong order?

\emph{Hint:} The answer is in Section~\ref{Sec_Reorder_Tags}.

% ================================================================

\subsection*{Ex-15-E: Bypassing}
\label{Ex-15-E-Bypassing}

(Please refer Section~\ref{Sec_Bypassing}.)

\begin{itemize}

 \item[(A)] Implement a scoreboard unit with bypassing/short-circuiting as
       described in the above note.

       \emph{Hint:} Needs BSV ``Concurrent Registers'' (Section~\ref{Sec_CRegs}).

 \item[(B)] What are the implications of bypassing/short-circuiting on the
       length of combinational paths in a design, and the consequent
       effect on achievable clock frequency?

\end{itemize}

% ================================================================

\subsection*{Ex-15-F: Store Buffers}
\label{Ex-15-F-Store-Buffers}

(Please refer Section~\ref{Sec_Store_Buffers}.)

\begin{itemize}

 \item[(A)] When the Retire stage sends a ``commit/discard'' message, how
       do we know that it applies to the pending STORE at the head of
       the store-buffer queue?

 \item[(B)] For a speculative LOAD, its value may come from memory or from
       the store-buffer.  In what order should it scan these sources?

 \item[(C)] Does the Retire stage need to send both ``commit'' and
       ``discard'' messages?  If requests are accompanied by an
       instruction-number, the Retire stage could only send ``commit''
       messages accompanied by the instruction-number.  The
       store-buffer can discard all pending STOREs with earlier
       instruction numbers.

       Discuss the implications of such a design on the size of the
       store-buffer.

 \item[(D)] Register reads/writes and memory reads/writes both raise
       similar ``hazards''.

       Explore a ``scoreboard''-like solution for memory accesses, and
       comment on the possibilities and difficulties for doing this.

       Explore a ``store-buffer''-like solution for register accesses,
       and comment on the possibilities and difficulties for doing
       this.

\end{itemize}

% ****************************************************************

\hdivider

\section*{Chapter 16: RISC-V: the Fife pipelined CPU code}

% ================================================================

\subsection*{Ex-16-A: Rule Atomicity}
\label{Ex-16-A-Register-Read-and-Dispatch}

(Please refer Section~\ref{Sec_mkRR_RW}.)

Consider this hypothetical scenario: suppose the \verb|stall|
condition is false.  Then, we need to dequeue \verb|f_Decode_to_RR|
and do one or more enqueues into \verb|f_RR_to_Retire| and
\verb|f_RR_to_EX_Control|, \verb|f_RR_to_EX_Int| or
\verb|f_DMem_S_req.enq|.  Is it possible that we perform the dequeue
and then are unable to perform the enqueue(s) because the
corresponding output FIFO happens to be full?

\emph{Hint:} rule atomicity

% ================================================================

\subsection*{Ex-16-B: Rule conditions}
\label{Ex-16-B-Rule-Conditions}

(Please refer Section~\ref{Sec_mkRR_RW}.)

Write a boolean expression representing the overall firing condition
for the rule.  Briefly: all FIFO-modifying actions (dequeue, enqueue)
have implicit conditions, but for each FIFO, that condition is only
relevant if the conditions on the surrounding if-then-else's select
that action.

% ================================================================

\subsection*{Ex-16-C: Debugging Stalled Rules}
\label{Ex-16-C-Debugging-Stalled-Rules}

(Please refer Section~\ref{Sec_mkRR_RW}.)

When debugging the implementation, it is useful to know if, due to
some coding mistake, rule \verb|rl_RR_Dispatch| is stalled forever.
For example, for some instruction with an rd, if the Retire stage did
not send back the rd's scoreboard-release, that register will be
forever ``busy''.

Add a register to count consecutive stalls, initially 0.  In the rule,
whenever we successfully dispatch an instruction, reset the counter to
0.  Whenever we stall, increment the stall counter, and if the
stall-counter reaches some chosen threshold value, prints debugging
messages and executes \verb|$finish| to terminate simulation.

% ================================================================

\subsection*{Ex-16-D: CSR method conflict}
\label{Ex-16-D-CSR-Method-Conflict}

(Please refer Section~\ref{Sec_Fife_Retire_CSRRxx}.)

The provided implementation of the CSR registers
(Section~\ref{Sec_RISCV_CSRs}) will in fact prevent the two methods
\verb|csrs.mav_csrrxx()| and \verb|csrs.mav_exception()| from being
invoked in a single action, {\ie} in a single rule, due to
method-ordering constraints.  Try this experiment: copy the body of
\verb|rl_exception| into \verb|rl_Retire_CSRRxx|; try compiling it
with \verb|bsc|, and observe the compiler's complaint message.

% ================================================================

\subsection*{Ex-16-E: Resolving a CSR method conflict}
\label{Ex-16-E-Resolving-a-CSR-Method-Conflict}

(Please refer Section~\ref{Sec_Fife_Retire_CSRRxx}.)

Modify the CSRs module method \verb|mav_csrrxx()| to combine the
actions of the original \verb|mav_csrrrxx()| with the actions of
\verb|mav_exception()|.  The result type will have to be extended
(say, a 3-tuple) to also return the trap-vector PC in case of an
exception.  Modify \verb|rl_Retire_CSRRxx| so that it immediately does
a Fetch-redirect with the trap-vector PC in case of an exception.
There is no longer any need to shift out of \verb|MODE_PIPE| into
\verb|MODE_EXCEPTION|.

This will save a cycle in handling CSRRxx instructions.  But what is
the impact on hardware complexity in the \verb|mkCSRs| module?

% ================================================================

\subsection*{Ex-16-F: ECALL/EBREAK}
\label{Ex-16-F-ECALL-EBREAK}

(Please refer Section~\ref{Sec_Retire_ECALL_EBREAK}.)

In the exercises in Section~\ref{Sec_Fife_Retire_CSRRxx}, we discussed
why, in \verb|rl_Retire_CSRRxx| we postponed exception handling to the
separate rule \verb|rl_exception|: \verb|rl_Retire_CSRRxx| itself
accesses the CSRs, and \verb|rl_exception| accesses the CSRs, and we
wanted to avoid contention.

\vspace{1ex}

Here, \verb|rl_Retire_ECALL_EBREAK| does not otherwise access the
CSRs, so there is no such contention, so why not handle the exception
directly here (copy the body of \verb|rl_exception| here) and save a
cycle?  Discuss the impact on hardware structure.  Try it, examine the
generated Verilog, and examine the performance impact (which will
depend on the RISC-V program being run, of course).

% ================================================================

\subsection*{Ex-16-G: Update Rd {\vs} Retire Store Buffer}
\label{Ex-16-G-Rd-vs-Store-Buffer}

(Please refer Section~\ref{Sec_Fife_Retire_DMem}.)

Inside \verb|rl_Retire_EX_DMem| we invoke ``\verb|fa_update_rd()|''
always, but we invoke ``\verb|f_retire_store_buf()|'' only when there
was no exception.  Justify this.

% ================================================================

\subsection*{Ex-16-H: Retiring Deferred DMem}
\label{Ex-16-H-Retiring Deferred DMem}

(Please refer Section~\ref{Sec_Fife_Retire_DMem_deferred}.)

This rule fields a memory response, but does not send any
commit/discard message to the store-buffer.  Why not?

% ****************************************************************

\hdivider

\section*{Chapter 17: {\BSV}: Rules and Methods II: Improved performance with CRegs (Concurrent Registers)}

% ================================================================

\subsection*{Ex-17-A: {\tt mkPipelineFIFOF}}
\label{Ex-17-A-PipelineFIFOs}

(Please refer Section~\ref{Sec_mkPipelineFIFOF}.)

The example code in the text for {\tt mkPipelineFIFOF} had a
particular choice of indexes for the dequeue-side methods, the
enqueue-side methods and for the clear method, respectively,
corresponding to the first row of the table below:

\begin{center}
\begin{tabular}{|c|c|c|}
\hline
    0  &  1  &  2 \\
\hline
  {\tt notEmpty}, {\tt first}, {\tt deq}  &
  {\tt notFull}, {\tt enq} &
  {\tt clear} \\
\hline
  {\tt notEmpty}, {\tt first}, {\tt deq} &
  {\tt clear} &
  {\tt notFull}, {\tt enq} \\
\hline
  {\tt clear} &
  {\tt notEmpty}, {\tt first}, {\tt deq} &
  {\tt notFull}, {\tt enq} \\
\hline
\end{tabular}
\end{center}

What changes in the behavioral semantics (ordering) of the FIFO if we
choose the indexes in the second or third row of the table instead
(moving the {\tt clear} method between the dequeue and enqueue sides,
or ahead of both of them?  \emph{Hint:} think about simultaneous
invocation of {\tt clear} with either an {\tt enq}, or a {\tt deq}, or
both.

% ================================================================

\subsection*{Ex-17-B: Composition of {\tt mkBypassFIFOF} and {\tt mkPipelineFIFOF}}
\label{Ex-17-B-Pipeline-FIFO-Composition}

(Please refer Section~\ref{Sec_mkBypassFIFOF_mkPipelineFIFOF}.)

What happens if we reverse the composition---use
\verb|mkPipelineFIFOF| upstream and \verb|mkBypassFIFOF| downstream?

\begin{itemize}

 \item How many ticks will it take for a datum to traverse the pair?

 \item Will \verb|rl_P| and \verb|rl_C| still be conflict-free (and
       therefore can fire in either logical order)?

 \item Are there any combinational paths through the pair?

 \item Consider combinational paths in rule \verb|rl_P| ending at the
       \verb|enq| method of the first FIFO, and combinational paths in
       rule \verb|rl_C| starting at the \verb|first| method of the
       second FIFO.  What is the difference between these when we
       compose \verb|mkBypassFIFOF| $\longrightarrow$ \verb|mkPipelineFIFOF|
       {\vs} \verb|mkPipelineFIFOF| $\longrightarrow$ \verb|mkBypassFIFOF|?

\end{itemize}

% ****************************************************************
% Use this only if this chapter/appendix has an odd number of pages

\input{blankpage}

% ****************************************************************

% \normalsize
