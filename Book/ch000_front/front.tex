% -*- mode: fundamental -*-

\pagestyle{empty}

\begin{center}

\vspace*{1.5in}

{\LARGE\bf Learn RISC-V CPU Implementation and BSV}

\vspace{1ex}

{\large\bf RISC-V: a major open-standard Instruction Set Architecture} \\
{\large\bf BSV: a modern, open-source High-Level Hardware Design Language}

\vspace{2cm}

{\Large Rishiyur S. Nikhil}

Bluespec, Inc.

\vspace*{0.5in}

\begin{figure}[htbp]
  \centerline{\includegraphics[height=1in,angle=0]{Figures/Bluespec_Logo_2022-10}}
\end{figure}

\vfill

\copyright{} 2023-2025 Rishiyur S.Nikhil

\vspace{0.5in}

{\bf DRAFT:} \today \\
{\bf Please do not circulate without the author's permission.}

\end{center}

% ----------------------------------------------------------------

\input{blankpage}

% ----------------------------------------------------------------

% \newpage

\vspace*{2in}

% ----------------------------------------------------------------

% \input{blankpage}

% ================================================================
% ACKNOWLEDGEMENTS AND PREFACE

\newpage

\pagenumbering{roman}

\vspace*{1.5cm}

\noindent
\subsection*{Acknowledgements: BSV}

The original ideas for a rule-based computation model for hardware
design were developed at MIT by James Hoe (currently Professor,
Carnegie Mellon University) and Arvind (Professor, MIT).  The ideas
behind CRegs (concurrent registers) were also developed later at MIT
by Daniel Rosenband and Arvind.

The idea of embedding this computation model in a Haskell-like
language (including types, typeclasses, static elaboration computation
model, higher-order programming, and monadic static elaboration) was
due to Lennart Augustsson.

The embedding in SystemVerilog-like syntax, technical refinement of
the formal semantics of BSV, extension to multiple clock domains and
implementation and improvements of the \emph{bsc} compiler is due to
several employees (including the author) of Sandburst, Inc. and
Bluespec, Inc. since 2003.

Thanks to all the employees of Bluespec, Inc. and to all the users of
BSV over the years--- commercial, academic and independent--- for
their feedback and insights on BSV and how to teach it.

Thanks to Bluespec, Inc., for agreeing to open-source BSV with its
\emph{bsc} compiler and libraries in 2020.

\noindent
\subsection*{Acknowledgements: RISC-V}

Thanks to the team from Universithy of California, Berkeley led by
Professor Krste {Asanovi\'c} for their tremendous gift to the world of
the free and open RISC-V specification, as a sophisticated,
industrial-strength, ISA.

Thanks to Bluespec, Inc., for supporting the author in creating
previous RISC-V CPU BSV designs (Flute, Piccolo, and Magritte), C
simulators for RISC-V (Cissr V1 and V2) and FPGA-based RISC-V Systems
(including Catamaran), all of which have greatly informed the new
designs Fife and Drum in this book.

% ----------------------------------------------------------------

% \input{blankpage}

% \noindent
% \section*{Preface}

% \vspace*{1cm}

% \noindent
% \emph{TBD: this book is still in draft form}

% ================================================================
% TABLE OF CONTENTS, SHORT

\newpage

{\small

\shorttoc{Short Table of Contents}{0}

\input{blankpage}

\newpage

}

% ================================================================
% TABLE OF CONTENTS, DETAILED

\newpage

\pagestyle{myheadings}

\markboth{CONTENTS}{}

\addcontentsline{toc}{chapter}{Detailed Table of Contents}

{\small

\renewcommand*\contentsname{Detailed Table Of Contents}

\tableofcontents

% ================================================================
% LIST OF FIGURES

\newpage

\addcontentsline{toc}{chapter}{List of Figures}

\listoffigures

\newpage

}

% ================================================================
% Use this only if table of contents has an odd number of pages

% \input{blankpage}

% ================================================================
