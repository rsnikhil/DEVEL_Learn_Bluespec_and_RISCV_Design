% -*- mode: fundamental -*-

% ****************************************************************

\chapter{Glossary}

\markboth{Ch \arabic{chapter}: Glossary (DRAFT)}{\copyrightnotice}

\setcounter{page}{1}
\renewcommand{\thepage}{\Alph{chapter}-\arabic{page}}

\label{apx_Glossary}

% ****************************************************************

\begin{itemize}

\item[\bf 2's Complement] See entry for ``Two's Complement''.

\item[\bf ACTs] Architecture Compatibility Tests.  A test suite under
  development by RVI to verify that a given implementation complies
  with a particular subset of the RISC-V ISA.  A candidate
  implementation must run the relevant ACTs and produce correct
  ``expected'' output signatures.

\item[\bf API] Application Programming Interface.  Term commonly used
  in many programming languages, methodologies and protocols to
  describe the set of functions/procedures/methods used to interact
  with a module/object by external entitities (from outside the
  module/object).  The API clearly separates external concerns from
  internal concerns.  External concerns are about ``what'' a method
  does or sequence of methods do: what are their argument and result
  types, and what do they (abstractly) achieve.  Internal concerns are
  about ``how'' methods do what they are supposed to do.  This
  separation of concerns also allow transparently substituting a
  module implementation with an alternate implementation ({\eg} for
  greater efficiency) without disturbing the external context.

\item[\bf ASIC] Application-Specific Integrated Circuit. A kind of
  electronic device that represents a desired digital circuit directly
  in silicon and has been fabricated for that purpose (not
  customizable and general-purpose like an FPGA).

\item[\bf BRAM] Block RAM.  A memory component in an FPGA, usually
  implemented with SRAM.

\item[\bf BSV, BH] An open-source, modern, High-Level HDL.  Two
  optional syntaxes (choose to one's taste): BSV has traditional
  Verilog-like syntax, BH has traditional Haskell-like syntax.

\item[\bf BTB] Branch Target Buffer.  A component of a PC-predictor
  module.  See Section~\ref{Sec_BTBs}.

\item[\bf CISC] Complex Instruction Set Computer (see also, for
  contrast, ``RISC'').  Refers to an ISA where a single instruction
  may express quite a bit of computation, such as memory accesses, ALU
  ops, and even iteration.

\item[\bf CPI] Cycles per Instruction.  Please see entry for its
  inverse, IPC.

\item[\bf CPU] Central Processing Unit.  The computational element of
  a computer.

\item[\bf CReg] Concurrent Register.  See Section~\ref{Sec_CRegs}.  Also
  known as EHRs (Ephemeral History Registers).

\item[\bf CSRs] Control and Status Registers.  These are special
  registers in the ISA, most of which are accessibly only while
  executing at higher privilege levels (Machine and Supervisor).
  Certain key CSRs play a central role in disciplined transition
  between privilege levels, in virtual memory, and in memory
  protection.

\item[\bf DRAM] Dynamic Random Access Memory.  A kind of silicon chip
  that implements memory.  Compared to SRAM, is larger (number of
  bits), denser (bits per silicon area), cheaper (\$ per bit), uses
  less power (watts per bit) and is more complex to operate (needing
  regular refreshing {\etc}). Usually off-chip (not part of an ASIC or
  an FPGA).

\item[\bf DUT] Design Under Test.  Term commonly used by hardware
  designers to indicate the artefact being designed, to contrast in
  particular with the ``testbench'' or ``test harness'' which
  surrounds the design during testing.  DUT is often articulated as a
  word in its own right (pronounced ``Dutt'', as in the famous name in
  Bombay cinema \url{https://en.wikipedia.org/wiki/Nargis_Dutt}).

\item[\bf EHR] Ephemeral History Register.  The original name for
  BSV's CRegs, when developed by Daniel Rosenband and Arvind at MIT
  \cite{RosenbandMEMOCODE04, Rosenband2005b}.

\item[\bf FPGA] Field Programmable Gate Array.  A kind of electronic
  device that has configurable circuits that can be customized to
  represent any desired digital design.  These are catalog parts
  available from several vendors.

\item[\bf FPGA Board] A circuit board containing one or more FPGAs, a
  power supply, and DRAM memories.  Often contains other facilities
  such as GPIO, UARTs, JTAG, PCIe bus connections, Ethernet
  connection, USB connection, Flash memory, and so on.

\item[\bf FSM] Finite State Machine.  A sequential process that moves
  (``transitions'') from one state to another in a fixed repertoire of
  states.  Transitions may loop back to earlier states, and may
  conditionally select one of a set of alternative next-states.

\item[\bf GPIO] General Purpose Input Output.  An electronic device
  attached to a computer system. When the CPU stores a byte/word to a
  GPIO address, the bits of the word appear as electronic signals from
  the device, and can be used as an \emph{actuator}---switch on/off a
  back of LED lamps, a relay, a motor, {\etc.}.  When the CPU loads a
  byte/word from a GPIO address, it can read the state of a
  \emph{sensor}---switches, photocells, motor speed, temperature,
  {\etc.}

\item[\bf GPR] General Purpose Register.  For RISC-V, just a synonym
  for the basic register set holding integers.  They are ``general
  purpose'' in the sense that software is free to use them in any way
  (in contrast with some earlier ISAs that restricted certain
  registers to certain roles, such as holding addresses).

\item[\bf HDL] Hardware Design Language.  A language in which one can
  represent circuits, and for which there are tools that can render a
  program into actual circuits for FPGAs and ASICs.  Examples include:
  BSV, BH, Chisel, Verilog, SystemVerilog, VHDL.

\item[\bf HLHDL] High-Level Hardware Design Language.  An HDL with
  higher-levels of abstraction and more powerful constructs and
  semantics compared to the traditional HDLs Verilog, SystemVerilog
  and VHDL, in the same sense that modern software programming
  languages (Java, Python, Javascript, Haskell, OCaml, ...) have
  higher-levels of abstraction than C/C++ which, in turn, have higher
  levels of abstraction than Assembly Language.  Examples include BSV,
  BH (the Haskell-syntax variant of BSV), Chisel, and HLS.

\item[\bf HLS] High Level Synthesis.  The term typically used for
  tools and methodology that compile C/C++/SystemC programs into
  hardware.  HLS can be fragile in that it works best only on certain
  subsets of C/C++ (``simple rectangular loop and array'' algorithms),
  and require certain coding styles and directives.

\item[\bf ILP] Instruction-Level Parallelism.  A measure of how many
  instructions can be executed in parallel without violating the
  canonical sequential (one-instruction-at-a-time) semantics of an
  ISA.

\item[\bf IPC] Instructions Per Clock (or its inverse, CPI, or clocks
  per instruction).  A component, together with clock speed (cycles
  per second), of a CPU's application performance.  Application
  performance depends on clock speed, IPC, and total number of
  instructions.

\item[\bf ISA] Instruction Set Architecture.  A specification of
  instructions: how an instruction is coded in bits; ``architectural
  state'' (PC, registers {\etc}); what it means to execute an
  instruction; assembly language syntax.  The specification is
  described independently of any particular implementation,
  traditionally in a manual with text and diagrams, occasionally and
  recently also in a formal-specification language.

  An ISA can (and typically does) have many possible implementations,
  varying widely in speed, size, power, cost, technology (ASIC, FPGA),
  {\etc} Examples of famous ISAs and vendors who supply
  implementations include RISC-V (diverse vendors), x86 (Intel and
  AMD), ARM (Arm, Apple, Samsung, others), Sparc (Sun, Oracle,
  Fujitsu, others), MIPS (MIPS, Inc.), Power and PowerPC (IBM,
  others), ...

\item[\bf Microarchitecture] The structural and behavioral details of
  an ISA implementation that are \emph{below} the level of abstraction
  of the ISA, {\ie} not demanded by the ISA but chosen by the
  implementor for practical reasons (speed, power, area, cost, ...).
  Examples: pipelines, branch prediction, scoreboards, register
  renaming, out-of-order execution, superscalarity, instruction
  fission and fusion, replicated execution units, store-buffers, ...

\item[\bf MMIO] Memory-Mapped Input-Output.  In RISC-V, the CPU reads
  and writes registers in a device using ordinary LOAD and STORE
  instructions.  The memory system interprets the addresses to direct
  such requests to a device.  Using LOADs and STOREs, the CPU can
  control the device, send data to the device and retrieve data from
  the device.

\item[\bf OOO] Out-Of-Order.  Refers to advanced CPU
  microarchitectures which replicate functional units (computational
  and memory access units); and allow each functional unit immediately
  to execute any instruction that is ``ready'' (because its inputs are
  available), even if that is not in program order.  Often combined
  with SuperScalarity, to achieve vastly more instruction-level
  parallelism (ILP) compared to a simple in-order pipeline.

\item[\bf OS] Operating System.  Can vary from small, embedded,
  real-time OSs such as FreeRTOS, to more capable embedded OSs like
  Zephyr, to secure micro-kernels like seL4, to full-featured OSs like
  Linux, Windows, MacOS, Solaris, AIX, {\etc}

\item[\bf PBT] Peanut Butter on Toast.  No, seriously: Property-Based
  Testing.  This was a technique that was pioneered by Haskell's
  \emph{QuickCheck} (\url{https://en.wikipedia.org/wiki/QuickCheck}),
  and facilities for which have now been replicated in numerous other
  programming languages.  It is an automated technique that repeatedly
  generates random stimulus for the DUT and, for each such input,
  tests one or more \emph{properties} which are formal logical
  predicates expressing some required correctness property of the DUT.
  The stimuli are typically generated in increasing size order, where
  ``size'' depends on the type of stimulus.  When a property failure
  is encountered, the system attempts to ``shrink'' the failing
  stimulus by randomly omitting parts of the stimulus, in the hope of
  finding a smaller stimulus that exhibits the same failure.  The
  results can be quite dramatic, often producing very small stimuli
  that trigger some bug.

\item[\bf PTW] Page Table Walk.  A function in systems supporting
  virtual memory, to translate virtual memory addresses into physical
  memory addresses (not described in this book).  Requires multiple
  memory references to descend a ``tree'' data structure called the
  Page Table.

\item[\bf RAM] Random Access Memory.  So called because successive
  addresses can be unrelated (in contrast with memory technologies
  that might only support, for example, sequential address accesses).

\item[\bf RAS] Return Address Stack.  A component of a PC-predictor
  module.  See Section~\ref{Sec_RAS}.  (Within the field of DRAM
  technology can also stand for Row Address Strobe.)

\item[\bf RISC] Reduced Instruction Set Computer (see also, for
  contrast, ``CISC'').  Refers to an ISA that separate memory-access
  instructions from computational instructions, and where each
  instruction is amenable to fast pipeline implementations.

\item[\bf RISC-V] A particular standard ISA.  Originated circa
  2008-2010 in research at University of California, Berkeley, and
  subsequently spun out (2010s) into an international non-profit
  consortium ``RISC-V International'' (RVI) headquartered in
  Switzerland (\url{https://riscv.org}).

  Unlike other well-known ISAs, the RISC-V ISA is an \emph{open}
  standard, {\ie} implementors do not need to pay any license fee in
  order to use the ISA, which is one of the factors behind its wide
  adoption by hundreds of vendors.

\item[\bf ROB] Reorder buffer.  An array into which each completed
  instruction is placed, at a specific index that represents its
  proper ISA order.  Each slot is initially empty, and becomes full
  when a completed instruction is placed there.  As separate process
  drains completed instructions from this array in sequential order.

\item[\bf RTOS] Real-Time Operating System.  A typically small
  operating sytem for small embedded systems.  Compared to, say,
  Linux,
  \begin{tightlist}
   \item It may not support multiple privilege levels (Machine, Supervisor, User).
   \item It may not support multiple processes, or a variable number of processes.
   \item It may not support virtual memory.
   \item It may not support memory protection across processes.
   \item It may only support a limited repertoire of devices.
  \end{tightlist}

\item[\bf RTL] Register-Transfer Level/Language.  This is a level of
  abstraction of describing hardware that assumes that the available
  primitive components are clocked registers and combinational
  circuits for multiplexers, and basic arithmetic and logic functions
  (adders, subtractors, boolean operations, shifters, {\etc}).

  This is a higher level of abstraction than AND/OR/XOR/NOT gates
  which, in turn, are a higher level of abstraction than transistors
  which, in turn, are a higher level of abstraction than silicon
  regions.  Each layer of abstraction is automatically compiled to a
  lower layer using various tools.

\item[\bf RVI] RISC-V International.  See entry for RISC-V.

\item[\bf Slack] A measure in digital circuits for the amount by which
  a combinational circuit exceeds the delay requirement for a target
  clock speed.  For example, if the target clock speed is 250 Mhz,
  {\ie} has a 4 ns period, and a combinational circuit between
  registers only takes 3.2 ns, then it is said to have a ``positive
  slack'' of 0.8 ns.

\item[\bf SoC] System-on-a-chip.  Refers to a complete computing
  system on a chip, including one or more CPUs (with MMUs and caches),
  shared caches, interconnects, DRAM interface, JTAG, accelerators and
  devices, {\etc}

\item[\bf SRAM] Static Random Access Memory.  A kind of silicon chip
  that implements memory.  See DRAM above for comparison.  Usually
  on-chip in an ASIC or an FPGA.

\item[\bf Superscalar].  Refers to advanced CPU microarchitectures
  which fetch and execute multiple instructions at a time (typically
  2, 4, 8).  Often combined with OOO (Out-of-order), to achieve vastly
  more instruction-level parallelism (ILP) compared to a simple
  in-order pipeline.

\item[\bf SystemVerilog] One of the major HDLs.  Originally created in
  the 2000s as a proper superset of Verilog (and thereby subsuming
  Verilog), and incorporating many features from VHDL; incorporated
  some modern features from object-oriented software programming
  languages (principally used in verification testbenches in
  simulation only); then an IEEE standard that has gone through
  several versions.  Can be used for both analog and digital circuits.
  Some features can only be used in simulation (a ``synthesizable
  subset'' can be rendered into hardware).

\item[\bf TCM] Tightly Coupled Memory. Usually an SRAM (Static RAM)
  used directly as the memory of the CPU, with no cache in front of
  it.

\item[\bf TLB] Translation Look-aside Buffer.  A component used in
  systems supporting virtual memory, to speed up translation of
  virtual memory addresses into physical memory addresses (not
  described in this book.)

\item[\bf Two's Complement] A particular representation of positive
  and negative integers in bits (binary) that makes it possible to
  perform both addition and subtraction using the same
  hardware. Wikipedia has a good discussion:
  \url{https://en.wikipedia.org/wiki/Two%27s_complement}

\item[\bf UART] Universal Asynchronous Receiver/Transmitter.  An
  electronic device attached to a computer system through which the
  CPU can read ASCII characters from a keyboard and send ASCII
  characters to a display screen.  Typically used for the main console
  of a computer system.

\item[\bf UVM] Universal Verification Methodology.  A standard
  methodology and technology in SystemVerilog for testbenches for
  verification. Exploits the ``object-orientd programming'' aspects of
  SystemVerilog for reusability. Wikipedia has an introduction
  (\url{https://en.wikipedia.org/wiki/Universal_Verification_Methodology}).
  There are dozens of tutorials, textbooks and UVM IP providers, both
  open-source and proprietary.

\item[\bf Verilog] One of the two grand old HDLs (the other is VHDL).
  Originally created in the 1980s; then an IEEE standard that has gone
  through several versions; then subsumed by SystemVerilog.  Can be
  used for both analog and digital circuits.  Some features can only
  be used in simulation (a ``synthesizable subset'' can be rendered
  into hardware).

\item[\bf VCD] Value Change Dump.  A file written out during
  simulation (Verilog simulation or {\BLUESIM} simulation) that contains
  a clock-time-stamped record of every change on every bus (bundle of
  wires) in the hardware design.  This file can then be viewed
  graphically in any waveform viewer.  Waveform viewers are bundled
  into most commercial simulators. Alternatively, \emph{gtkwave} is a
  popular free and open-source waveform viewer.

\item[\bf VHDL] One of the two grand old HDLs (the other is Verilog).
  Originally created in the 1980s; then an IEEE standard that has gone
  through several versions. Many features were adopted by
  SystemVerilog.  Can be used for both analog and digital circuits.
  Some features can only be used in simulation (a ``synthesizable
  subset'' can be rendered into hardware).

\end{itemize}

% ****************************************************************
