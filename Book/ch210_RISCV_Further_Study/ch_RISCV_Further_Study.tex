% -*- mode: fundamental -*-

% ****************************************************************

\chapter{RISC-V: Suggested further study}

\markboth{Ch \arabic{chapter}: RISC-V: further study}{\copyrightnotice}

\setcounter{page}{1}
% \renewcommand{\thepage}{\arabic{page}}
\renewcommand{\thepage}{\arabic{chapter}-\arabic{page}}

\label{ch_RISCV_further_study}

% ****************************************************************

% ----------------
\vspace{2ex}

\centerline{\includegraphics[width=1in,angle=0]{Figures/Fig_Under_Construction}}

\vspace{2ex}
% ----------------

% ****************************************************************

\section{Introduction}

In this chapter we provide suggestions for further study of RISC-V
topics.  Each of these can also be seen as suggested exercises for the
interested student.

% ****************************************************************

\section{Implementing RV64I instead of RV32I}

% ****************************************************************

\section{M Extension}

Integer Multiply and Divide.

% ****************************************************************

\section{F and D Extensions}

IEEE single- and double-precision floating point.

% ****************************************************************

\section{C Extension}

Compressed instructions

Implementation: wholly in Decode stage.  Can be shared between Fife and Drum.

Affect of compressed instructions on Fetch.

Affect of compressed instructions on PC-prediction.

Affect of non-32-bit alignment of compressed instructions.

% ****************************************************************

\section{A extention}

AMO operations

LR, SC, AMOxxx instructions and their implementation in the memory system.

% ****************************************************************

\section{Advanced branch prediction}

Is a form of online machine-learning (past history is the ``training data'').

Branch instruction taken/not-taken hints.

Hysteresis in prediction.

Branch-Target buffers (BTBs)

Return-Address Stacks (RASs)

% ****************************************************************

\section{Register renaming: towards out of order processing}

An alternative to the counter in the scoreboard.

Have more phys regs than 32.  Maintain a table that dynamically maps
logical register num to phys reg num.  This is "virtualizing" the
registers.

In S3, for each instruction, allocate a new phys register num for its
Rd, use that phys num in the instr going forward, and update the
log-phys map.

In case of trap or misprediction, have to restore the logical-to-physical map.

% ****************************************************************

\section{Advanced bypassing: towards dataflow and out-of-order processing}

In S3, do not stall any instruction for register-hazards, just forward
each instruction into its appropriate S4 pipe, with each input
register either having its value or a marker saying "not yet available".

In each S4 pipe, stall if the head of the queue does not yet have its
input values.

In each S4 pipe, as soon as an output register value is ready,
broadcast it to the other S4 pipes, which update any entries awaiting
that register value.  This may release the stalling of the head entry,
allowing it to execute.

Without register renaming, we have to keep the "counter" from the
scoreboard.  With register renaming, there is no need, since each
physical register will have only one outstanding writer.

In each exec unit, we can treat the pending instrs as a set, not FIFO.
I.e., execute any instr that is "ready".  Then we have OOO dataflow
processing.

% ****************************************************************

\section{Memory systems: TCMs, Caches, PMPs}

Separate I- and D-caches.

FENCE.I: for ``manual'' I- and D-Mem coherence.

FENCE: flushing caches for for devices.

Multi-level caches and cache hierarchies.

Non-blocking caches.

Cache-coherence.

Memory Protection with PMPs.

% ****************************************************************

\section{Memory Systems: Virtual Memory}

Page Tables, TLBs, Virtual Memory.

% ****************************************************************

\section{Performance measurement}

CSRs TIME, MCYCLE.

Other ``hpmcounter'' CSRs for other events.
Counter enables.

% ****************************************************************

\section{Testing}

ISA tests. 

Tandem Verification

Sail formal model.

ACTs (sp ?)

% ****************************************************************

\section{Interrupts}

\begin{tightlist}
  \item General concepts: CSRs MIP and MIE; minimal MSTATUS with interrupt-enable bits

  \item Interrupts are initially disabled using the
        MSTATUS.interrupt-enable bit immediately; CSRxx can be used to
        re-enable.

  \item MMIO addresses MTIME, MTIMECMP.

  \item Interrupts are handled just like traps; the only question is:
        when to check for interrupts and respond.

  \item How does MIE bit return to 0?

\end{tightlist}

PLIC, CLIC

Interrupt/trap delegation.

% ****************************************************************

\section{Linux and server-class capability}

Multiple privilege levels: Machine, Supervisor, User

\subsection{Hypervisor support}

% ****************************************************************

\subsection{RISC-V ISA Formal Specification}

Sail model

% ****************************************************************
